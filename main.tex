\documentclass{article}
\usepackage{graphicx} 
\usepackage{booktabs}
\usepackage{array}
\usepackage{amsmath}
\usepackage{subcaption}

\graphicspath{{figures/}}
\title{Experimental Study of Group Effects in Tsunami Debris Impact and Damming Forces}
\date{September 2025}

\begin{document}

\maketitle

\section{Introduction}
\begin{itemize}
    \item Tsunami-driven debris fields impose both short-duration impact loads and sustained damming loads on coastal infrastructure.
    \item Predicting these forces is challenging because they depend on highly nonlinear fluid--structure--debris interactions.
    \item Debris characteristics that influence loading include:
    \begin{itemize}
        \item Orientation (longitudinal, transverse, random; see Figs.~\ref{fig:configurations} and \ref{fig:configurations_random}).
        \item Size distribution (single-size vs multi-size).
        \item Surface area density (ratio of tightly packed debris area to debris spread area).
    \end{itemize}

\end{itemize}

\begin{figure}[htbp]
    \centering
    \includegraphics[width=0.8\textwidth]{Configurations.jpg}
    \caption{Regular configurations.}
    \label{fig:configurations}
\end{figure}

\begin{figure}[htbp]
    \centering
    \includegraphics[width=0.48\textwidth]{configurations_rand.jpg}
    \caption{Random configurations.}
    \label{fig:configurations_random}
\end{figure}

\section{Testing Methodology}
\begin{itemize}
    \item Experiments were conducted in a tsunami flume equipped with a rigid mid-flume structure.
    \item Two wave types were tested: unbroken solitary waves (primary focus) and broken solitary waves. This paper focuses on the unbroken solitary wave tests. 
    \item Debris fields were generated in three configurations:
    \begin{itemize}
        \item Regular longitudinal (aligned with flow).
        \item Regular transverse (perpendicular to flow).
        \item Random (both single-size and multi-size).
    \end{itemize}
    \item Force records were decomposed using Butterworth filters:
    \begin{itemize}
        \item High-pass filtering to isolate short-duration impact peaks (Fig.~\ref{fig:high_low_pass}).
        \item Low-pass filtering to capture long-duration damming forces (Fig.~\ref{fig:high_low_pass}).
    \end{itemize}
\end{itemize}

\begin{figure}[htbp]
    \centering
    \includegraphics[width=0.8\textwidth]{high_low_pass.png}
    \caption{Force decomposition using Butterworth filters.}
    \label{fig:high_low_pass}
\end{figure}

\begin{figure}[h!]
    \centering
    \begin{subfigure}{0.48\textwidth}
        \includegraphics[width=\linewidth]{Reg_Lift_U_1_L_D__Masters_NHERIDeprisImpact2_goodtests_Reg_Lift_U_1_L_Trial04_Peak.png}
        \caption{1L debris.}
    \end{subfigure}
    \hfill
    \begin{subfigure}{0.48\textwidth}
        \includegraphics[width=\linewidth]{Reg_Lift_U_24_L_D__Masters_NHERIDeprisImpact2_goodtests_Reg_Lift_U_24_L_Trial04_Peak.png}
        \caption{24L debris.}
    \end{subfigure}
    \caption{Force time histories for peak impacts under unbroken wave trials.}
    \label{fig:timehist_unbroken_peaks}
\end{figure}

\begin{figure}[h!]
    \centering
    \begin{subfigure}{0.48\textwidth}
        \includegraphics[width=\linewidth]{Reg_Lift_U_1_L_D__Masters_NHERIDeprisImpact2_goodtests_Reg_Lift_U_1_L_Trial04_Damming.png}
        \caption{1L debris.}
    \end{subfigure}
    \hfill
    \begin{subfigure}{0.48\textwidth}
        \includegraphics[width=\linewidth]{Reg_Lift_U_24_L_D__Masters_NHERIDeprisImpact2_goodtests_Reg_Lift_U_24_L_Trial04_Damming.png}
        \caption{24L debris.}
    \end{subfigure}
    \caption{Force time histories for damming under unbroken wave trials.}
    \label{fig:timehist_unbroken_damming}
\end{figure}

    
\end{figure}
\section{Impact Force Identification}
\begin{itemize}
    \item Video evidence and filtered signals revealed that debris impact occurs in two stages:
    \begin{itemize}
        \item First impact: initial strike on the unsubmerged structure (Fig.~\ref{fig:first_impact}).
        \item Later impacts: submerged strikes caused by overtopping, suction, and recirculation (Fig.~\ref{fig:second_impact}).
    \end{itemize}
    \item This behavior was consistent across debris types and configurations, highlighting the need to distinguish between first and late peaks.
\end{itemize}
\begin{figure}[htbp]
    \centering
    \includegraphics[width=0.6\textwidth]{first_impact.jpg}
    \caption{Initial unsubmerged strike.}
    \label{fig:first_impact}
\end{figure}

\begin{figure}[htbp]
    \centering
    \includegraphics[width=0.6\textwidth]{second_impact.jpg}
    \caption{Later submerged strike.}
    \label{fig:second_impact}
\end{figure}

\section{Results: Impact Forces (Unbroken Wave)}

\subsection{Regular Configurations}
\begin{itemize}
    \item First impact peaks increased linearly with debris number (Fig.~\ref{fig:firstpeak_regular_remap}).
    \item Later impacts showed nonlinear growth, saturating at higher debris counts (Fig.~\ref{fig:laterpeak_regular_remap}).
    \item Longitudinal layouts produced the highest impact forces, while transverse layouts yielded smaller peaks due to water cushioning.
    \item For the 24T and 16T2 tests, debris field area exceeded the structure area. Only part of the debris contacted the structure, so the data was remapped to reflect the effective area (Figs.~\ref{fig:firstpeak_regular_remap}, \ref{fig:laterpeak_regular_remap}).
\end{itemize}

\begin{figure}[htbp]
    \centering
    \includegraphics[width=0.8\textwidth]{FirstPeak_Regular_RemappedT.png}
    \caption{First peak forces in regular arrangements (remapped).}
    \label{fig:firstpeak_regular_remap}
\end{figure}

\begin{figure}[htbp]
    \centering
    \includegraphics[width=0.8\textwidth]{LaterPeak_Regular_RemappedT.png}
    \caption{Later peak forces in regular arrangements (remapped).}
    \label{fig:laterpeak_regular_remap}
\end{figure}

\subsection{Random Configurations}
\begin{itemize}
    \item Random fields produced scattered and smaller peaks compared to regular ones (Fig.~\ref{fig:random_peaks_first}). 
    \item Multi-size fields yielded lower impacts because smaller blocks carried less momentum (Figs.~\ref{fig:random_peaks_first}, \ref{fig:boxplot_8}, \ref{fig:boxplot_16}, \ref{fig:boxplot_24}).
    \item Surface area density effects:
    \begin{itemize}
        \item Increasing density dispersed debris more widely.
        \item At higher counts (16, 24), spreading reduced the chance of simultaneous impacts.
        \item For 8 debris pieces, density differences were minimal.
        \item For 16 and 24 pieces, increasing density correlated with stronger impacts (Figs.~\ref{fig:boxplot_16}, \ref{fig:boxplot_24}).
    \end{itemize}
\end{itemize}
   
\begin{figure}[htbp]
    \centering
    \includegraphics[width=0.8\textwidth]{First_Peak_Random_Single_vs_Multi_ByDensityGradient.png}
    \caption{First peak forces in random fields.}
    \label{fig:random_peaks_first}
\end{figure}

\begin{figure}[htbp]
    \centering
    \includegraphics[width=0.8\textwidth]{Boxplot_Density_vs_PeakValues_8Debris.png}
    \caption{First peaks, 8 pieces.}
    \label{fig:boxplot_8}
\end{figure}

\begin{figure}[htbp]
    \centering
    \includegraphics[width=0.8\textwidth]{Boxplot_Density_vs_PeakValues_16Debris.png}
    \caption{First peaks, 16 pieces.}
    \label{fig:boxplot_16}
\end{figure}

\begin{figure}[htbp]
    \centering
    \includegraphics[width=0.8\textwidth]{Boxplot_Density_vs_PeakValues_24Debris.png}
    \caption{First peaks, 24 pieces.}
    \label{fig:boxplot_24}
\end{figure}

\section{Results: Damming Forces (Unbroken Waves)}
\begin{itemize}
    \item Damming forces, extracted by low-pass filtering, reflected sustained blockage (Figs.~\ref{fig:damming_regular_remap}, \ref{fig:damming_random}).
    \item Single-size debris: increasing density led to stable jams and higher damming loads.
    \item Multi-size debris: smaller blocks prevented stable blockages, keeping forces lower (Fig.~\ref{fig:damming_random}).
    \item Longitudinal and transverse cases produced similar damming magnitudes, showing geometry was less critical than density (Fig.~\ref{fig:damming_regular_remap}).
\end{itemize}

\begin{figure}[htbp]
    \centering
    \includegraphics[width=0.8\textwidth]{Damming_Regular_L_T_SplitByTrial_Remapped.png}
    \caption{Damming forces in regular arrangements.}
    \label{fig:damming_regular_remap}
\end{figure}

\begin{figure}[htbp]
    \centering
    \includegraphics[width=0.8\textwidth]{Damming_Random_Single_vs_Multi_ByDensityGradient.png}
    \caption{Damming forces in random fields.}
    \label{fig:damming_random}
\end{figure}

% \section{Flow Velocities}
% The mean flow velocity in the flume was confirmed to be XYZ. This section is reserved for later analysis.

\section{Conclusions}
\begin{itemize}
    \item Longitudinal debris groups generated the largest impact peaks, while transverse groups were dominated by damming loads.
    \item Random and multi-size fields produced smaller and more variable forces.
    \item Surface area density strongly influenced impact probability once debris spread exceeded structure width.
    \item For 24T and 16T2 trials, partial contact required remapping.
    \item First impacts scaled with debris count under aligned conditions, while later impacts saturated at higher counts.
    \item Multi-size fields weakened damming loads due to disrupted jamming.
\end{itemize}

\section{Limitations}
\begin{itemize}
    \item Due to high impact forces, loading could not be recorded at the front of the flume. All measured forces are reaction forces and include structural vibrations.
    \item Flume geometry amplified some effects, such as water cushioning, as flow was obstructed by the side walls.
    \item The relative width of debris and structure limited the representativeness of some tests.
    \item Practical limits on debris quantity constrained the range of tested cases.
\end{itemize}

\section{References}
Bonus, J., Arduino, P., Motley, M., and Eberhard, M. (2022). ``Multi-Scale Numerical Simulation of Tsunami-Driven Debris-Field Impacts.'' In \textit{Ports 2022}, 328--39. https://doi.org/10.1061/9780784484395.033.

Derschum, C., Nistor, I., Stolle, J., and Goseberg, N. (2018). ``Debris Impact under Extreme Hydrodynamic Conditions Part 1: Hydrodynamics and Impact Geometry.'' \textit{Coastal Engineering}, 141: 24--35. https://doi.org/10.1016/j.coastaleng.2018.08.016.

Mascarenas, D., Motley, M. R., Eberhard, M. O., Arduino, P., and Serrone, A. (2022). ``Quantifying and Understanding Structural Loading from Wave-Driven Debris Fields.'' In \textit{Ports 2022}, 523--34. https://doi.org/10.1061/9780784484395.052.

Shekhar, K., Winter, A. O., Alam, M. S., Arduino, P., Miller, G. R., Motley, M. R., Eberhard, M. O., Barbosa, A. R., Lomonaco, P., and Cox, D. T. (2020). ``Conceptual evaluation of tsunami debris field damming and impact forces.'' \textit{Journal of Waterway, Port, Coastal, and Ocean Engineering}, 146(6), 04020039.

\end{document}
